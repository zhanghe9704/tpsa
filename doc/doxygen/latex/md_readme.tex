\subsection*{Announcement}

T\+HE S\+O\+F\+T\+W\+A\+RE IS P\+R\+O\+V\+I\+D\+ED \char`\"{}\+A\+S I\+S\char`\"{}, W\+I\+T\+H\+O\+UT W\+A\+R\+R\+A\+N\+TY OF A\+NY K\+I\+ND, E\+X\+P\+R\+E\+SS OR I\+M\+P\+L\+I\+ED, I\+N\+C\+L\+U\+D\+I\+NG B\+UT N\+OT L\+I\+M\+I\+T\+ED TO T\+HE W\+A\+R\+R\+A\+N\+T\+I\+ES OF M\+E\+R\+C\+H\+A\+N\+T\+A\+B\+I\+L\+I\+TY, F\+I\+T\+N\+E\+SS F\+OR A P\+A\+R\+T\+I\+C\+U\+L\+AR P\+U\+R\+P\+O\+SE A\+ND N\+O\+N\+I\+N\+F\+R\+I\+N\+G\+E\+M\+E\+NT. IN NO E\+V\+E\+NT S\+H\+A\+LL T\+HE A\+U\+T\+H\+O\+RS OR C\+O\+P\+Y\+R\+I\+G\+HT H\+O\+L\+D\+E\+RS BE L\+I\+A\+B\+LE F\+OR A\+NY C\+L\+A\+IM, D\+A\+M\+A\+G\+ES OR O\+T\+H\+ER L\+I\+A\+B\+I\+L\+I\+TY, W\+H\+E\+T\+H\+ER IN AN A\+C\+T\+I\+ON OF C\+O\+N\+T\+R\+A\+CT, T\+O\+RT OR O\+T\+H\+E\+R\+W\+I\+SE, A\+R\+I\+S\+I\+NG F\+R\+OM, O\+UT OF OR IN C\+O\+N\+N\+E\+C\+T\+I\+ON W\+I\+TH T\+HE S\+O\+F\+T\+W\+A\+RE OR T\+HE U\+SE OR O\+T\+H\+ER D\+E\+A\+L\+I\+N\+GS IN T\+HE S\+O\+F\+T\+W\+A\+RE.

\subsection*{About this code}

This code allows users to do computations using Truncated Power Series Algebra (T\+P\+SA) and/or Differential Algebra (DA).

For T\+P\+SA and DA, please refer to chapter 8 in \href{http://inspirehep.net/record/595287/files/slac-pub-9574.pdf}{\texttt{ {\itshape Lecture Notes on Special Topics in Accelerator Physics}}} by Prof. Alex Chao and chapter 2 in \href{http://bt.pa.msu.edu/cgi-bin/display.pl?name=AIEP108book}{\texttt{ {\itshape Modern Map Methods in Particle Beam Physics}}} by Prof. Martin Berz.

This code is developed based on Dr. Lingyun Yang\textquotesingle{}s tpsa codes in C++ . His codes (\mbox{\hyperlink{tpsa_8cpp}{tpsa.\+cpp}} and \mbox{\hyperlink{tpsa_8h}{tpsa.\+h}}) are included in this repository. They are untouched, except for a few functions that are commented off and replaced by functions in \mbox{\hyperlink{tpsa__extend_8cc}{tpsa\+\_\+extend.\+cc}}. Please get permission from Dr. Lingyun Yang before you redistribute \mbox{\hyperlink{tpsa_8cpp}{tpsa.\+cpp}} and \mbox{\hyperlink{tpsa_8h}{tpsa.\+h}}.

The major change is the memory management. The current memory management works in the following way. Before any T\+P\+S\+A/\+DA calculation, one needs to reserve the memory for \$n\$ T\+PS vector. A memory pool is allocated in the heap, which can be considered as \$n\$ slots, each for one T\+PS vector. A linked list is created to reference the available (unused) slots. Two pointers point to the beginning and the end of the linked list respectively. When a new T\+PS vector is created, the first slot will be assigned to it and the beginning pointer points to the following slot in the linked list. When a T\+PS vector goes out of its scope, the slot will be reset to empty and liked to the end of the list. The end pointer will be updated and point to the new end.



A new data type \mbox{\hyperlink{struct_d_a_vector}{D\+A\+Vector}} is created as a wrapper of the T\+PS vector. The following mathematical operator and functions are overloaded for \mbox{\hyperlink{struct_d_a_vector}{D\+A\+Vector}}, so that a variable of \mbox{\hyperlink{struct_d_a_vector}{D\+A\+Vector}} type can be used as a intrinsic type in calculations.

Math operator overloaded

\tabulinesep=1mm
\begin{longtabu}spread 0pt [c]{*{3}{|X[-1]}|}
\hline
\PBS\centering \cellcolor{\tableheadbgcolor}\textbf{ Left hand  }&\PBS\centering \cellcolor{\tableheadbgcolor}\textbf{ Operator  }&\PBS\centering \cellcolor{\tableheadbgcolor}\textbf{ Right hand   }\\\cline{1-3}
\endfirsthead
\hline
\endfoot
\hline
\PBS\centering \cellcolor{\tableheadbgcolor}\textbf{ Left hand  }&\PBS\centering \cellcolor{\tableheadbgcolor}\textbf{ Operator  }&\PBS\centering \cellcolor{\tableheadbgcolor}\textbf{ Right hand   }\\\cline{1-3}
\endhead
\PBS\centering \mbox{\hyperlink{struct_d_a_vector}{D\+A\+Vector}}  &\PBS\centering +  &\PBS\centering \mbox{\hyperlink{struct_d_a_vector}{D\+A\+Vector}}   \\\cline{1-3}
\PBS\centering double  &\PBS\centering +  &\PBS\centering \mbox{\hyperlink{struct_d_a_vector}{D\+A\+Vector}}   \\\cline{1-3}
\PBS\centering \mbox{\hyperlink{struct_d_a_vector}{D\+A\+Vector}}  &\PBS\centering +  &\PBS\centering double   \\\cline{1-3}
\PBS\centering &\PBS\centering +  &\PBS\centering \mbox{\hyperlink{struct_d_a_vector}{D\+A\+Vector}}   \\\cline{1-3}
\PBS\centering \mbox{\hyperlink{struct_d_a_vector}{D\+A\+Vector}}  &\PBS\centering -\/  &\PBS\centering \mbox{\hyperlink{struct_d_a_vector}{D\+A\+Vector}}   \\\cline{1-3}
\PBS\centering \mbox{\hyperlink{struct_d_a_vector}{D\+A\+Vector}}  &\PBS\centering -\/  &\PBS\centering double   \\\cline{1-3}
\PBS\centering double  &\PBS\centering -\/  &\PBS\centering \mbox{\hyperlink{struct_d_a_vector}{D\+A\+Vector}}   \\\cline{1-3}
\PBS\centering &\PBS\centering -\/  &\PBS\centering \mbox{\hyperlink{struct_d_a_vector}{D\+A\+Vector}}   \\\cline{1-3}
\PBS\centering \mbox{\hyperlink{struct_d_a_vector}{D\+A\+Vector}}  &\PBS\centering $\ast$  &\PBS\centering \mbox{\hyperlink{struct_d_a_vector}{D\+A\+Vector}}   \\\cline{1-3}
\PBS\centering \mbox{\hyperlink{struct_d_a_vector}{D\+A\+Vector}}  &\PBS\centering $\ast$  &\PBS\centering double   \\\cline{1-3}
\PBS\centering double  &\PBS\centering $\ast$  &\PBS\centering \mbox{\hyperlink{struct_d_a_vector}{D\+A\+Vector}}   \\\cline{1-3}
\PBS\centering \mbox{\hyperlink{struct_d_a_vector}{D\+A\+Vector}}  &\PBS\centering /  &\PBS\centering \mbox{\hyperlink{struct_d_a_vector}{D\+A\+Vector}}   \\\cline{1-3}
\PBS\centering \mbox{\hyperlink{struct_d_a_vector}{D\+A\+Vector}}  &\PBS\centering /  &\PBS\centering double   \\\cline{1-3}
\PBS\centering double  &\PBS\centering /  &\PBS\centering \mbox{\hyperlink{struct_d_a_vector}{D\+A\+Vector}}   \\\cline{1-3}
\PBS\centering \mbox{\hyperlink{struct_d_a_vector}{D\+A\+Vector}}  &\PBS\centering =  &\PBS\centering \mbox{\hyperlink{struct_d_a_vector}{D\+A\+Vector}}   \\\cline{1-3}
\PBS\centering \mbox{\hyperlink{struct_d_a_vector}{D\+A\+Vector}}  &\PBS\centering =  &\PBS\centering double   \\\cline{1-3}
\PBS\centering \mbox{\hyperlink{struct_d_a_vector}{D\+A\+Vector}}  &\PBS\centering +=  &\PBS\centering \mbox{\hyperlink{struct_d_a_vector}{D\+A\+Vector}}   \\\cline{1-3}
\PBS\centering \mbox{\hyperlink{struct_d_a_vector}{D\+A\+Vector}}  &\PBS\centering +=  &\PBS\centering double   \\\cline{1-3}
\PBS\centering \mbox{\hyperlink{struct_d_a_vector}{D\+A\+Vector}}  &\PBS\centering -\/=  &\PBS\centering \mbox{\hyperlink{struct_d_a_vector}{D\+A\+Vector}}   \\\cline{1-3}
\PBS\centering \mbox{\hyperlink{struct_d_a_vector}{D\+A\+Vector}}  &\PBS\centering -\/=  &\PBS\centering double   \\\cline{1-3}
\PBS\centering \mbox{\hyperlink{struct_d_a_vector}{D\+A\+Vector}}  &\PBS\centering $\ast$=  &\PBS\centering \mbox{\hyperlink{struct_d_a_vector}{D\+A\+Vector}}   \\\cline{1-3}
\PBS\centering \mbox{\hyperlink{struct_d_a_vector}{D\+A\+Vector}}  &\PBS\centering $\ast$=  &\PBS\centering double   \\\cline{1-3}
\PBS\centering \mbox{\hyperlink{struct_d_a_vector}{D\+A\+Vector}}  &\PBS\centering /=  &\PBS\centering \mbox{\hyperlink{struct_d_a_vector}{D\+A\+Vector}}   \\\cline{1-3}
\PBS\centering \mbox{\hyperlink{struct_d_a_vector}{D\+A\+Vector}}  &\PBS\centering /=  &\PBS\centering double   \\\cline{1-3}
\end{longtabu}


Math functions overloaded\+:


\begin{DoxyItemize}
\item sqrt
\item exp
\item log
\item sin
\item cos
\item tan
\item asin
\item acos
\item atan
\item sinh
\item cosh
\item tanh
\item pow
\item abs
\item erf
\end{DoxyItemize}

Some test results for efficiency are presented in the following. They are done in a Windows 10 desktop with Intel Xeon (R) E5-\/1620 processor at 3.\+60 G\+Hz. Table 1 shows the time cost for composition of one D\+A/\+T\+PS vector of six bases with six D\+A/\+T\+PS vectors. First column shows the order of the vectors, second column the number of terms in each vector, third column the time using the DA data type with revised memory management, and the fourth column the time using the original code. Table 2 shows the time of composition of six DA vectors, each having six bases, with the other group of six DA vectors. The composition in group cost less time if compared with separate compositions.

Table 1. Time (in second) of composition

\tabulinesep=1mm
\begin{longtabu}spread 0pt [c]{*{4}{|X[-1]}|}
\hline
\PBS\centering \cellcolor{\tableheadbgcolor}\textbf{ Order  }&\PBS\centering \cellcolor{\tableheadbgcolor}\textbf{ No. of terms  }&\PBS\centering \cellcolor{\tableheadbgcolor}\textbf{ DA  }&\PBS\centering \cellcolor{\tableheadbgcolor}\textbf{ T\+P\+SA   }\\\cline{1-4}
\endfirsthead
\hline
\endfoot
\hline
\PBS\centering \cellcolor{\tableheadbgcolor}\textbf{ Order  }&\PBS\centering \cellcolor{\tableheadbgcolor}\textbf{ No. of terms  }&\PBS\centering \cellcolor{\tableheadbgcolor}\textbf{ DA  }&\PBS\centering \cellcolor{\tableheadbgcolor}\textbf{ T\+P\+SA   }\\\cline{1-4}
\endhead
2  &28  &\$7.\+57\textbackslash{}times 10$^\wedge$\{-\/6\}\$  &\$6.\+25\textbackslash{}times 10$^\wedge$\{-\/6\}\$   \\\cline{1-4}
4  &210  &\$7.\+50\textbackslash{}times 10$^\wedge$\{-\/4\}\$  &\$1.\+44\textbackslash{}times 10$^\wedge$\{-\/2\}\$   \\\cline{1-4}
6  &924  &\$4.\+48 \textbackslash{}times 10$^\wedge$\{-\/3\}\$  &\$8.\+39\textbackslash{}times 10$^\wedge$\{-\/2\}\$   \\\cline{1-4}
8  &3003  &\$9.\+90 \textbackslash{}times 10$^\wedge$\{-\/1\}\$  &\$2.\+55\$   \\\cline{1-4}
10  &8008  &\$15.\+49\$  &\$44.\+60\$   \\\cline{1-4}
\end{longtabu}


Table 2. Time (in second) of group composition

\tabulinesep=1mm
\begin{longtabu}spread 0pt [c]{*{2}{|X[-1]}|}
\hline
\PBS\centering \cellcolor{\tableheadbgcolor}\textbf{ Order  }&\PBS\centering \cellcolor{\tableheadbgcolor}\textbf{ DA   }\\\cline{1-2}
\endfirsthead
\hline
\endfoot
\hline
\PBS\centering \cellcolor{\tableheadbgcolor}\textbf{ Order  }&\PBS\centering \cellcolor{\tableheadbgcolor}\textbf{ DA   }\\\cline{1-2}
\endhead
2  &\$1.\+51\textbackslash{}times 10$^\wedge$\{-\/5\}\$   \\\cline{1-2}
4  &\$1.\+04\textbackslash{}times 10$^\wedge$\{-\/3\}\$   \\\cline{1-2}
6  &\$4.\+42\textbackslash{}times 10$^\wedge$\{-\/2\}\$   \\\cline{1-2}
8  &\$1.\+05\$   \\\cline{1-2}
10  &\$16.\+04\$   \\\cline{1-2}
\end{longtabu}


More information is available at doc/doxygen/html/index.\+html.

\subsection*{How to compile and use this code}

You will need a C++ compiler that supports C++ 11 standard. There are three ways to use the code as follows\+:


\begin{DoxyItemize}
\item Download the source files. Include \char`\"{}tpsa\+\_\+extend.\+h\char`\"{} and \char`\"{}da.\+h\char`\"{} in your project and compile.
\item The code is developed using Code\+::\+Blocks I\+DE. There are two C\+::B profiles under the cbp directory\+: tpsa\+\_\+lib.\+cbp and tpsa\+\_\+dll.\+cbp for static library and dynamic library respectively. The cbp files are tested in Windows 10 with gcc compiler.
\item You can also use cmake to compile the code into both a static library and a dynamic library. This is only tested in Ubuntu 16.\+04.

{\ttfamily cmake .} {\ttfamily make}

Here is an example of compiling the code under Ubuntu 16.\+04.

Assume I have cloned the codes to the following folder\+:

\$\+H\+O\+ME/tpsa

Inside the above folder, run\+:

{\ttfamily cmake .}

The Makefile will be generated.

Then run\+:

{\ttfamily make}

Both the static lib and the shared lib of tpsa will be generated. You can find the following two files\+:

libtpsa.\+a and libtpsaso.\+so

Now you can use any of them to compile your file. Here let us compile the \mbox{\hyperlink{examples_8cc}{examples/examples.\+cc}} using libtpsa.\+a.

{\ttfamily gcc \mbox{\hyperlink{examples_8cc}{examples/examples.\+cc}} -\/o tpsa\+\_\+exp -\/I ./include/ -\/L ./lib -\/ltpsa -\/lstdc++ -\/lm -\/std=c++11}

You can also use libtpsaso.\+so.

{\ttfamily gcc \mbox{\hyperlink{examples_8cc}{examples/examples.\+cc}} -\/o tpsa\+\_\+exp -\/std=c++11 -\/Iinclude -\/L. -\/ltpsaso -\/lstdc++ -\/lm}

The executable file tpsa\+\_\+exp will be generated.

To run the tpsa\+\_\+exp file, tell the OS where to find the libtpsaso.\+so\+:

{\ttfamily export L\+D\+\_\+\+L\+I\+B\+R\+A\+R\+Y\+\_\+\+P\+A\+TH=\$\+L\+D\+\_\+\+L\+I\+B\+R\+A\+R\+Y\+\_\+\+P\+A\+TH\+:\$\+H\+O\+ME/tpsa}

Run the executable\+:

{\ttfamily ./tpsa\+\_\+exp}

The result is shown as follows\+: V \mbox{[}16\mbox{]}              Base  \mbox{[} 4 / 286 \mbox{]}
\end{DoxyItemize}





1.\+000000000000000e+00      0  0  0     0 

-\/1.\+000000000000000e+00      1  0  0     1   

2.\+000000000000000e+00      0  1  0     2   

5.\+000000000000000e-\/01       0  0  1     3

......

\subsection*{Acknowledgement}

Thanks to Dr. Lingyun Yang for providing his tpsa code.

\subsection*{Contact}

Contact the author by hezhang.\+A\+T.\+jlab.\+org. 